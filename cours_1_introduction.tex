Dans ce premier cours on a commencé en plenum avec les étudiants des 3 ateliers numériques. On a eu quelques informations générales sur le numérique, donc ce qui ce fait dans les écoles au secondaire 1 et au secondaire 2. Au secondaire 2 il n'y pas vraiment d'unification des plans de cours. ILs ont parlé des deux majeurs problèmes aux quelles les enseignants font face, d'un côté l'utilisation de l'IA de la part des élèves et donc l'évaluation. D'un autre côté avec les classes BYODs, il est difficile de faire en sorte que les élèves sont bien entrain de travailler et pas de faire d'autres activités sur leurs appareils numériques. 
Une fois en groupe par atelier on a fait un tour de table pour connaitre les connaissances et utilisation actuelles des étudiants de la classe. Les niveau d'utilisations sont très variés. Ensuite un tour administratif sur l'organisation du cours. 

Questions/Reflexion:

- Serais-ce bien d'indiquer à mes élèves les ressources qui sont générés par IA?
- Est-ce que je vais utiliser Google Sites pour le site ou faire un site en html?
- Est-ce que c'est une mauvaise chose de télécharger des films illégalement pour en utiliser des extraits étant donné que par les autres voies c'est plus difficile d'avoir accès à la bande vidéo brute? 